%%%%%%%%%%%%%%%%%%%%%%%%%%%%%%%%%%%%%%%%%%%%%%%%%%%%%%%%%%%%%%%%%%%%%%%%%%%%%
% Seitenlayout
%%%%%%%%%%%%%%%%%%%%%%%%%%%%%%%%%%%%%%%%%%%%%%%%%%%%%%%%%%%%%%%%%%%%%%%%%%%%%

% Aufbau von Labels
% 						XX_XX_XX_XX_XX_XX_XX_XX_XX
% 						chapter_
% 						section_
% 						subsection_
% 						subsubsection_
% 						subsubsection_
%						subsubsubsection_
%						subsubsubsubsection_
% 						Abbildung_
%						Unterabbildung_
% 						Tabelle

% Kopfzeile
\ihead{}
\chead{}
\ohead{\leftmark}

% Fußzeile
\ifoot{}
\cfoot{\pagemark}
\ofoot{}

% Kapitelname
\chapter{Geschichte}
\label{21_00_00_00_00_00_00_00_00_00}

% Standard Kopf- und Fußzeile für erste Seite des Kapitels verwenden
\thispagestyle{plain}


%%%%%%%%%%%%%%%%%%%%%%%%%%%%%%%%%%%%%%%%%%%%%%%%%%%%%%%%%%%%%%%%%%%%%%%%%%%%%
% Inhalt
%%%%%%%%%%%%%%%%%%%%%%%%%%%%%%%%%%%%%%%%%%%%%%%%%%%%%%%%%%%%%%%%%%%%%%%%%%%%%

Die Geschichte der heutigen Hochschule Würzburg-Schweinfurt reicht bis 1807 zurück und ist verknüpft mit den Vorgängereinrichtungen \glqq Balthasar-Neumann-Polytechnikum\grqq~des Bezirks Unterfranken, der \glqq Höheren Wirtschaftsschule\grqq~der Stadt Würzburg und der Werkkunstschule der Stadt Würzburg.

% Vertikaler Abstand
\vskip 10mm

\section{Ganz früher}
\label{21_01_00_00_00_00_00_00_00_00}

Am 1. August 1971 nahm die aufgrund des Bayerischen Fachhochschulgesetzes (FHG) gegründete Fachhochschule Würzburg-Schweinfurt mit 1566 Studierenden in sieben Studiengängen den Betrieb auf. Im Laufe der Zeit wurden das Angebot um die Studiengänge Soziale Arbeit (1972), Kunststofftechnik und Vermessung (1973), Informatik (1975), Pflegemanagement (1995), Betriebswirtschaft (1998), Wirtschaftsinformatik (2000), Medienmanagement (2001), Ingenieurinformatik (2003), Logistik (2008) ergänzt.

% Vertikaler Abstand
\vskip 10mm

\subsection{In Würzburg}
\label{21_01_01_00_00_00_00_00_00_00}
Am Standort Würzburg wurden die Fachrichtungen Architektur, Bauingenieurwesen, Betriebswirtschaft und Grafikdesign (heute Kommunikationsdesign) angeboten.

% Vertikaler Abstand
\vskip 10mm

\subsection{In Schweinfurt}
\label{21_01_02_00_00_00_00_00_00_00}

% \begin{wrapfigure}[Zeilen]{Position}[Überhang]{Breite}
% \includegraphics[Größe]{Bild}
% \end{wrapfigure}

\begin{wrapfigure}[9]{l}{0.51\textwidth}
    \includegraphics[height=0.22\textwidth, width=0.5\textwidth,angle=0]
    	{Rundbau.jpg}
    \caption 	[Rundbau in Schweinfurt]
    				{Rundbau}
    \label{20_01_02_00_00_00_00_01_00_00}
\end{wrapfigure}


In Schweinfurt die Fachrichtungen Elektrotechnik, Maschinenbau und Wirtschaftsingenieurwesen. In Abbildung \ref{20_01_02_00_00_00_00_01_00_00} ist der Rundbau in Schweinfurt zu sehen. Von 1991 bis 2000 baute man im Auftrag des Wissenschaftsministeriums eine neue Abteilung in Aschaffenburg auf. Dort wurde zunächst der Studiengang Betriebswirtschaft angeboten, ab 1997 auch der Studiengang Elektrotechnik. Im Jahr 2000 ging aus der damaligen Abteilung Aschaffenburg eine eigenständige Fachhochschule hervor.
Am 14. Februar 2003 legte der damalige Wissenschaftsminister Hans Zehetmeier den Grundstein für ein knapp 14,5 Millionen Euro teures Hörsaalgebäude für die Abteilung Schweinfurt. Im Keller des Gebäudes finden sich moderne Computersäle. Im Erdgeschoss bietet eine großzügige Aula Platz für Veranstaltungen. Kreisförmig um diesen Versammlungsraum befinden sich Dekanate, Büros und der größte Hörsaal des Gebäudes. Im ersten und zweiten Stock des von Stahlbeton und Aluminium dominierten Baus finden sich klimatisierte und multimedial ausgestattete Hörsäle.

% Vertikaler Abstand
\vskip 10mm

\subsection{Heute}
\label{21_01_03_00_00_00_00_00_00_00}
Nach der Einweihung durch Wissenschaftsminister Thomas Goppel am 4. Oktober 2004, findet in dem zirka 3000 Quadratmeter großen Rundbau ein Großteil des Lehrbetriebs in der Abteilung Schweinfurt statt. Die älteren Gebäude am Standort Schweinfurt wurden weitestgehend geräumt. Es finden dort seit 2004 umfangreiche Renovierungsmaßnahmen statt.
2006 hat die Hochschule Würzburg-Schweinfurt die Namensrechte des größten Hörsaals der Hochschule (Abteilung Würzburg) an Aldi-Süd vermietet. Die Sparkasse Würzburg hat sich 2006 als \glqq Sponsor\grqq~eines weiteren Hörsaals zur Verfügung gestellt. Kurze Zeit später wurden folgende weitere Hörsaalsponsoren gefunden:

\begin{itemize}
\item SALT Solutions (Abt. Würzburg)
\item Fränkische Rohrwerke Königsberg (Abt. Würzburg)
\item Fresenius Medical Care (Abt. Schweinfurt)
\item LEONI (Abt. Schweinfurt)
\item Warema-Renkhoff (Abt. Schweinfurt)
\end{itemize}

% Vertikaler Abstand
\vskip 10mm

Im Jahr\footnote{Das Jahr ist eine Zeitdauer, die je nach Definition eine unterschiedliche Länge hat. Der derzeit weltweit übliche und – insbesondere in Europa – gesetzlich gültige Kalender ist der gregorianische Kalender (vgl. \cite{Theis.2018,Vijayakumaran.2016}).} wurde der Name \glqq Fachhochschule Würzburg-Schweinfurt\grqq~in \glqq Hochschule für angewandte Wissenschaften – Fachhochschule Würzburg-Schweinfurt\grqq~geändert. Seit 1. Mai 2011 führt die Hochschule den offiziellen Namen\ldots

\begin{center}
\glqq Hochschule für angewandte Wissenschaften Würzburg-Schweinfurt\grqq
\end{center}
 
